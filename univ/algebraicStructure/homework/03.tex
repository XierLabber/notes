\documentclass[UTF8]{ctexart}
\usepackage{amsmath}
\usepackage{forest}
\usepackage{algorithm}
\usepackage{algorithmic}
\usepackage{tabularx}
\usepackage{booktabs}
\title{\vspace{-4cm}2023代组第三次作业}
\author{2000013058 杨仕博}
\date{\today}
\begin{document}
\maketitle

\section*{半群与独异点}

\subsection*{11}

记$V/\sim =<A/\sim, \bigotimes >$,则

\begin{tabular}{ccc}
    \hline
    $\bigotimes$ & [a] & [b]\\
    \hline
    [a] & [a] & [b]\\
    \hline
    [b] & [b] & [a]\\
    \hline
\end{tabular}

\subsection*{12}

(1)

首先,我们证明R是等价关系:

这是因为,$x = x\Rightarrow xRx$(自反性)

$xRy\Rightarrow x = y\lor (x\in I\land y\in I)
\Rightarrow y = x\lor (y\in I\land x\in I)
\Rightarrow yRx$(对称性)

$xRy, yRz\Rightarrow (x = y\land y = z)\lor
(x\in I\land y\in I\land y = z)\lor
(x = y\land y\in I\land z\in I)\lor
(x \in I\land y\in I\land z\in I)
\Rightarrow
(x = z)\lor
(x\in I\land z\in I)\lor
(x\in I\land z\in I)\lor
(x\in I\land z\in I)\lor
\Rightarrow
xRz$(传递性)

其次,若$xRy, sRt$:

若$x = y \land s = t$,则
$x\circ s = y\circ t\Rightarrow (x\circ s)R(y\circ t)$

若$x = y\land s\in I\land t\in I$,则
$x\circ s\in IS\Rightarrow x\circ s\in I$,
同理$y\circ t\in I$,故$(x\circ s)R(y\circ t)$

同理,若$s = t\land x\in I\land y\in I$,则
$(x\circ s)R(y\circ t)$

若$s\in I\land t\in I\land x\in I\land y\in I$,
同上有$x\circ s\in I, y\circ t\in I$,
$(x\circ s)R(y\circ t)$

故R是V上的同余关系

(2)

由$[x] = [y]\Leftrightarrow xRy\Leftrightarrow x = y
\lor(x\in I\land y\in I)$知
$S/R = \{I\}\bigcup \{\{x\} | x\notin I\}$

对于$x\in S-I, y\in S-I, \{x\}\bar\circ \{y\} = \{x\circ y\}$,
其他情况下$\{x\}\bar \circ \{y\} = I$

\section*{群}

\subsection*{2}

这个代数系统:

非空:由于G非空,因而G关于$\circ$构成的代数系统非空

封闭:$\forall a, b\in G, a\circ b = au^{-1}b\in G$

满足结合律:$\forall a, b, c\in G, a\circ (b\circ c) = 
au^{-1}(bu^{-1}c) = au^{-1}bu^{-1}c = a\circ b\circ c$

单位元存在:$u\in G, \forall a\in G, a\circ u = au^{-1}u = a, 
u\circ a = uu^{-1}a = a$

逆元存在:$\forall a\in G, a\circ (ua^{-1}u) = au^{-1}ua^{-1}u = u, 
(ua^{-1}u)\circ a = (ua^{-1}u)u^{-1}a = u$

因而是群

\subsection*{6}

\[
\begin{aligned}
    &(ab)^2 = abab = aabb\\
    \Rightarrow &a^{-1}ababb^{-1} = a^{-1}aabbb^{-1}\\
    \Rightarrow &ba = ab
\end{aligned}
\]

\subsection*{9(2)}

\[
\begin{aligned}
    &(ab)^r = e\\
    \Rightarrow &a(ba)^{r-1}b = e\\
    \Rightarrow &(ba)^{r-1} = a^{-1}b^{-1} = (ba)^{-1}\\
    \Rightarrow &(ba)^r = e
\end{aligned}    
\]

因此,$|ba|\Big| |ab|$,
同理,$|ab|\Big| |ba|$,因此,$|ab| = |ba|$

\subsection*{9(3)}

由上题,

$|abc| = |a(bc)| = |bca| = |b(ca)| = |cab|$

\subsection*{11}

首先我们断言:G中存在不为e的元素——否则,G中只有e一个
元素,G是交换群,矛盾!

任取G中不为e的元素a,那么$a^2\neq a$(否则,$a = e$),
这样,令$b = a^2\neq a$,有$ab = aaa = a^2a = ba$

\end{document}