\documentclass[UTF8]{ctexart}
\usepackage{amsmath}
\usepackage{forest}
\usepackage{algorithm}
\usepackage{algorithmic}
\usepackage{tabularx}
\title{\vspace{-4cm}2023代组第二次作业}
\author{2000013058 杨仕博}
\date{\today}
\begin{document}
\maketitle

\subsection*{29}

(1)

$\forall x\in Z, \phi(\Delta x) = \phi(x + 1) = (x + 1)\mod 2 = (x\mod 2 + 1)\mod 2 = \bar{\Delta} \phi(x)$

因而,$\phi$是$V_1$到$V_2$的同态

(2)

对于$\phi$引出的等价关系$\sim $,
\[
\begin{aligned}
    x\sim y &\Leftrightarrow \phi(x)\sim \phi(y)\\
        &\Leftrightarrow x\equiv y\mod 2
\end{aligned}    
\]

因而$\phi$导出的划分是$\{2Z+1, 2Z\}$

\subsection*{30}

(1)

由$\forall x\in A_k, nx\geq m\Rightarrow nx\in A_m$并且
$\forall x, y\in A_k, \phi(x + y) = n(x + y) = nx + ny = \phi(x) + \phi(y)$
知$\phi$是$V_1$到$V_2$的同态

(2)

对于$a, b\in V_1$,

\[
\begin{aligned}
    a\sim b &\Leftrightarrow \phi(a) = \phi(b)\\
        &\Leftrightarrow na = nb\\
        &\Leftrightarrow a = b\lor n = 0
\end{aligned}    
\]

故

\[
V_1 / \sim  = 
\begin{cases}
    <\{\{x\} | x\in Z \land x\geq k\}, \bigoplus >, 
    \{x\} \bigoplus\{y\} = \{x + y\} &(n\neq 0)\\
    <\{A_k\}, \bigoplus>, A_k \bigoplus A_k = A_k &(n=0)
\end{cases}
\]

\subsection*{7}

由$(a*b)*(a*b) = a*(b*a)*b = a*(a*b)*b = (a*a)*(b*b) = a*b$
知$a*b$是幂等元。

\subsection*{10}

(1)

对于$V$的子半群$W$,

若$3\in W$,则$3\bigotimes 3 = 1\in W$,此时若$2\in W$,
则$2\bigotimes 2 = 0\in W, W=Z_4$;若$2\notin W\land 0\in W$,
则$W=\{1, 3, 0\}$,其中元素的乘积均在$W$中,从而构成子半群;
若$2\notin W\land 4\notin W$,则$W = \{1, 3\}$,
其中元素的乘积均在$W$中,从而构成子半群。

若$3\notin W$,同样的,若$2\in W$,
则$2\bigotimes 2 = 0\in W, W=\{0, 1, 2\}\lor W = \{0, 2\}$,
上述两种可能中W的任两个(可以相同)的元素的积均在W中,构成子半群;
若$2\notin W$,则$W=\{0\}\lor W=\{1\}\lor W=\{0, 1\}$,
上述三种可能中W的任两个(可以相同)的元素的积均在W中,构成子半群

因此,$W = \{1, 3, 0\}$或$W = \{1, 3\}$或$W = \{0, 1, 2\}$或$W = \{0, 2\}$ 或
$W=\{0\}$或$W=\{1\}$或$W=\{0, 1\}$

(2)

在上述子半群中,
有1在的W或者只有一个元素的W必然是子独异点,剩余的$W = \{0, 2\}$
中任何两数相乘皆不为2,因此不是子独异点。

因此对于子独异点W,
$W = \{1, 3, 0\}$或$W = \{1, 3\}$或$W = \{0, 1, 2\}$ 或
$W=\{0\}$或$W=\{1\}$或$W=\{0, 1\}$

\end{document}