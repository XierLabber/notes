\documentclass[UTF8]{ctexart}
\usepackage{amsmath}
\usepackage{forest}
\usepackage{algorithm}
\usepackage{algorithmic}
\usepackage{tabularx}
\usepackage{booktabs}
\title{\vspace{-4cm}2023代组第七次作业}
\author{2000013058 杨仕博}
\date{\today}
\begin{document}
\maketitle

\subsection*{53}

由于$|\phi(H)|\Big||H|$(因为$\phi(H)$同构于H的某个商群),

$|\phi(H)|\Big||G_2|$(因为$\phi(H)$是$G_2$的子群),

故$|\phi(H)|\Big|(|H|, |G_2|)\Rightarrow |\phi(H)| = 1\Rightarrow \phi(H) = \{e_2\}$(其中$e_2$是$G_2$中的单位元)

故$H\subseteq ker(\phi)$

\subsection*{61}

首先,若$\phi$是自同构,那么$\forall x, y \in G, xy = \phi(x^{-1})\phi(y^{-1}) = \phi(x^{-1}y^{-1}) = yx$,
故G是交换群

其次,若G是交换群,那么,

$\forall x, y\in G, \phi(x) = \phi(y)\Rightarrow x^{-1} = y^{-1}\Rightarrow x = y$,故$\phi$是单射

$\forall x\in G, \phi(x^{-1}) = x$,故$\phi$是满射,

$\forall x, y\in G, \phi(xy) = y^{-1}x^{-1} = x^{-1}y^{-1} = \phi(x)\phi(y)$,故$\phi$是自同态,

因而,$\phi$是自同构

\subsection*{62}

(1)

记$I = \{0, 1, ..., n-1\}$

$\forall i, j\in I, \phi_t(a^ia^j) = a^{t(i + j)} = a^{ti}a^{tj} = \phi_t(a^i)\phi_t(a_j)$

故$\phi_t$是G的自同态

(2)

若$\phi_t$是自同构,那么$\exists i\in I, s.t.\ \phi(a^i) = a\Rightarrow \exists i\in I, s.t.\ it\mod n = 1\Rightarrow (n, t) = 1$

若$(n, t) = 1$,由裴蜀定理,$\exists i_0\in I, s.t.\ i_0t\mod n = 1$

因为$\forall i, j\in I, \phi_t(a^i) = \phi_t(a^j)\Leftrightarrow a^{it} = a^{jt}\Leftrightarrow it\equiv jt(\mod n)\Leftrightarrow ii_0t\equiv ji_0t(\mod n)\Leftrightarrow j\equiv i(\mod n)\Leftrightarrow a^i = a^j$,
故$\phi_t$为单射

又因为$\forall i\in I, \phi_t(a^{ii_0}) = a^{ii_0t} = a^i$,故$\phi_t$为满射。

因此,$\phi_t$是自同构。

\subsection*{68}

考察如下函数:

$f:G\rightarrow G/H\times G/K, \forall x\in G, $
$f(x) = (xH, xK)$

我们证明这是一个单射+同态映射,这样,G即和$f(G)$($G/H\times G/K$的子群)
同构。

由于$\forall x, y\in G, f(x) = f(y)\Rightarrow (xH, xK) = (yH, yK)\Rightarrow y^{-1}x\in H\land y^{-1}x\in K\Rightarrow y^{-1}x\in H\bigcap K\Rightarrow y^{-1}x = e\Rightarrow x = y$,
因而f是单射

又$\forall x, y\in G, f(xy) = (xyH, xyK) = (xH, xK)\times (yH, yK) = f(x)f(y)$
因而f是同态映射

故命题得证。

\subsection*{5}

我们先证明(2):

$a + a = a + a + a + a +(-a) +(-a) = a^2 + aa + aa + a^2 +(-a) +(-a) = (a + a)^2 +(-a) +(-a) = a + a +(-a) +(-a) = 0$

接下来我们证明(1):

$x + y = (x + y)^2 = x^2 + xy + yx + y^2 = x + xy + yx + y\Rightarrow xy + yx = 0$

又$xy + xy = 0$,故$xy = yx$

最后我们证明(3):

反设R是整环,由于$|R| > 3$,R中必然有和零元与幺元不同的元素,设为x,
由上有$x(x - 1) = x^2 - x = x - x = 0$,故$x = 0\lor x - 1 = 0\Rightarrow x = 0\lor x = 1$,与x不为零元或幺元矛盾

故命题得证

\subsection*{7}

(1)

由于n不是素数,可以设$n = pq, p\neq 1\land q\neq 1$,这样$1 < p, q < n$,又$pq = qp = 0$,故而$Z_n$中有零因子

(2)

若$(r, n) = d\neq 1$,则$r\times (n / d) = n\times (r / d) = 0, (n / d)\times r = 0$,故r为零因子

故r不是零因子$\Rightarrow (r, n) = 1$

若$(r, n) = 1$,由裴蜀定理知$\exists s\in Z, s.t.\ rs\equiv 1(\mod n)$,
若此时有$\exists t\in Z, s.t.\ rt = 0$,则$t = srt = 0$,故r不为左零因子,同理r
不为右零因子,故r不为零因子

命题得证

(3)

由(2),$Z_{18}$中的全部零因子为$2, 3, 4, 6, 8, 9, 10, 12, 14, 15, 16$

\subsection*{12}

首先证明n为素数。若n不为素数,则可设$n = pq, p, q\neq 1$,有$0 = ne = (pe)(qe)$,与F没有零因子矛盾!

之后,由于$\forall s\in\{1, 2, ..., n - 1\} C_n^s = (n / s) C_{n - 1}^{s - 1}$,该式展开分母中没有n的倍数(因为所有数都比n小),分子中有n,
故$n|C_n^s$,故$(a + b)^n = \sum\limits_{i = 0}^n C_n^i a^i b^{n - i} = a^n + b^n$

\end{document}