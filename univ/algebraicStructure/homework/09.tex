\documentclass[UTF8]{ctexart}
\usepackage{amsmath}
\usepackage{forest}
\usepackage{algorithm}
\usepackage{algorithmic}
\usepackage{tabularx}
\usepackage{booktabs}
\title{\vspace{-4cm}2023代组第九次作业}
\author{2000013058 杨仕博}
\date{\today}
\begin{document}
\maketitle

\subsection*{2}

(1) 不是格,因为[3, 4] = 12不在该集合中。

(2) 是格,因为这个集合是12的所有正因子集合,构成12
的正因子格。(一个数的两个因子的最大公因数和最小公倍数
都是这个数的因子)

(3) 是格,因为这个集合是36的所有正因子集合,构成36
的正因子格。(一个数的两个因子的最大公因数和最小公倍数
都是这个数的因子)

(4) 是格,因为$\forall i, j\in N, i\leq j, (5^i, 5^j) = 5^i$,
$[5^i, 5^j] = 5^j$都在这个集合中。

\subsection*{8}

(以下集合对应以他为载体的格)

$L_1$

3元子格:$\{b, c, e\}, \{b, d, e\}, \{a, b, e\}, \{a, b, c\}, \{a, b, d\}, \{a, c, e\}, \{a, d, e\}$

4元子格:$\{e, c, b, a\}, \{e, d, b, a\}, \{e, c, d, b\}$

5元子格:$L_1$

$L_2$

写程序枚举得

3元子格:

\{d, e, g\}, 
\{e, b, g\}, 
\{e, a, g\}, 
\{f, d, g\}, 
\{f, a, g\}, 
\{f, c, g\}, 
\{d, a, g\}, 
\{a, b, g\}, 
\{c, a, g\}, 
\{d, e, a\}, 
\{e, a, b\}, 
\{f, d, a\}, 
\{f, c, a\}

4元子格:

\{e, d, g, f\}
\{e, a, g, f\}
\{e, a, d, g\}
\{e, b, a, g\}
\{e, c, a, g\}
\{a, d, g, f\}
\{b, a, g, f\}
\{c, a, g, f\}
\{b, a, d, g\}
\{a, c, d, g\}
\{b, c, a, g\}
\{e, a, b, d\}
\{a, c, d, f\}

5元子格:

\{a, g, d, f, e\}
\{b, a, g, f, e\}
\{c, a, g, f, e\}
\{b, a, g, d, e\}
\{c, a, g, d, e\}
\{c, b, a, g, e\}
\{b, a, g, d, f\}
\{c, a, g, d, f\}
\{c, b, a, g, f\}
\{c, b, a, g, d\}

\subsection*{19}

G的子群格为

$L(G) = \{<a>, <a^p>, <a^{p^2}>, ..., <a^{p^t}>\}$

记L从最小元到最大元依次为$b_0, b_1, ..., b_t$

考察如下映射:$\phi(b_i) = <a^{p^i}>, \forall b_i\in L$(1)

那么,$\phi$是单射,因为象的子群相同则该子群的a的最低幂次的元素相同,
意味者b的脚标唯一固定,映射前的脚标一样进而输入相同。

$\phi$是满射,因为在(1)式中$a^{p^i}, i = 0, 1, 2, ..., t$都跑遍了,
进而所有元素都在象中。

$\phi$是同态,因为$\forall i, j \in \{0, 1, 2, ..., t\}, j\leq i$,
$\phi(b_i\land b_j) = \phi(b_j) = <a^{p^j}> = <a^{p^j}>\bigcap <a^{p^i}> = \phi(b_i)\bigcap \phi(b_j)$,
类似地$\phi(b_i\lor b_j) = \phi(b_i) = <a^{p^i}> = <a^{p^j}>\bigcup <a^{p^i}> = \phi(b_i)\bigcup \phi(b_j)$

故而二者同构。

\subsection*{28}

注:证明中的元素均为B中的任意元素

首先证明$<B, \oplus >$构成Abel群。

这是因为,由于B是布尔代数,故$\oplus$中每一次运算都封闭,进而$\oplus$
是封闭的。

$a\oplus b = (a\land \bar b) \lor (\bar a \land b) = (\bar a \land b) \lor (a\land \bar b) = (b\land \bar a) \lor (\bar b\land a) = b\oplus a$
,故满足交换律

\[
\begin{aligned}
    a\oplus(b\oplus c) &= (a\land \overline{(b\land \bar c)\lor (\bar b\land c)}) \lor (\bar a \land ((b\land \bar c)\lor (\bar b\land c)))\\
    &= (a\land \overline{b\land \bar c}\land \overline{\bar b\land c})\lor (\bar a\land b\land \bar c)\lor (\bar a\land \bar b\land c)\\
    &= (a\land(\bar b\lor c)\land (b\lor \bar c))\lor (\bar a\land b\land \bar c)\lor (\bar a\land \bar b\land c)\\
    &= (a\land \bar b\land \bar c)\lor (a\land c\land b)\lor (\bar a\land b\land \bar c)\lor (\bar a\land \bar b\land c)\\
    &= (a\land c\land b)\lor (\bar a\land \bar b\land c)\lor (\bar a\land b\land \bar c)\lor (a\land \bar b\land \bar c)
\end{aligned}
\]

类似地,$a\oplus b\oplus c = (a\land c\land b)\lor (\bar a\land \bar b\land c)\lor (\bar a\land b\land \bar c)\lor (a\land \bar b\land \bar c)$,
故$\oplus$满足结合律

又$a\oplus 0 = (a\land 1)\lor (\bar a\land 0) = a\lor 0 = a$

$a\oplus a = (a\land \bar a)\lor (\bar a\land a) = 0\lor 0 = 0$

故B中存在单位元和逆元

进而$<B, \oplus >$构成Abel群

接下来,由于$\land$封闭且满足结合律,故$<B, \otimes>$构成半群

由于$a\otimes(b\oplus c) = a\land ((b\land \bar c)\lor (\bar b \land c)) = (a\land b\land \bar c)\lor (a\land \bar b\land c)$

\[
\begin{aligned}
    (a\otimes b)\oplus (a\otimes c) &= ((a\land b)\land \overline{a\land c}) \lor (\overline{a\land b} \land (a\land c))\\
    &= (a\land b\land \bar c)\lor (a\land \bar b\land c)
\end{aligned}
\]

故分配律成立

因而$<B, \oplus, \otimes>$构成环

最后,由于$a\otimes a = a\land a = a$,故$<B, \oplus, \otimes>$构成布尔环

\subsection*{3}

在$7, 77, ..., 77...7$(共N+1个7)中,必然有两项模N同余,他们的差
必然形如$77...700...0$,为N的倍数,得证

\subsection*{5}

令$x_i$为第i天复习的小时数(i = 1,2,...,37),$S_l = \sum\limits_{i = 1}^lx_i$

考察$S_1, S_2, ..., S_37, S_1 + 13, S_2 + 13, ..., S_37 + 13$这74个数,
由于这些数中每一项均大于等于0,小于等于60 + 13 = 73,故而必然有两项
一样,又明显前一半数中两两不同,后一半数中两两不同,因而必然是$\exists
i, j\in\{1, 2, ..., 37\}, i < j, s.t. S_j = S_i + 13, S_j - S_i = 13$
,故第i+1天到第j天该生学习了13小时

\subsection*{9}

反设不然,那么放球数最少的盒子里至少有0个球,第二少的盒子里至少有1个球
,第k少的盒子里至少有k - 1个球(否则前k少的盒子中球的数目种类少于k,必然
有两个盒子球数相当)

所以总球数至少为$0 + 1 + 2 + ... + (n - 1) = \frac{n(n-1)}{2}$
,矛盾!

\subsection*{10}

由于所有数的和为$36\times 37 / 2 = 666$,故随机将圆盘12等分,必有一个扇形上数的和不少于$\lceil 666 / 12\rceil = 56$,这个扇形上有3个数字,得证

\end{document}