\documentclass[UTF8]{ctexart}
\usepackage{amsmath}
\usepackage{forest}
\usepackage{algorithm}
\usepackage{algorithmic}
\usepackage{tabularx}
\usepackage{booktabs}
\title{\vspace{-4cm}2023代组第五次作业}
\author{2000013058 杨仕博}
\date{\today}
\begin{document}
\maketitle

\subsection*{21}

熟知循环群的子群仍为循环群,因而$H_1, H_2$为循环群,
设$H_1 = <a>, H_2 = <b>, G = <g>, a = g^s, b = g^t$,
有$ab = g^{s + t} = ba$,又有$(r, s) = 1$,因而由熟知结论,
$<ab>$为$rs$阶循环群。而$<ab>\subseteq G$,故$G = <ab> \subseteq H_1H_2$。

又对任意$x=a^p\in H_1, y=b^q\in H_2$,由中国剩余定理,
存在$l\in Z^+,\ s.t.l\equiv p(\mod r), l\equiv q(\mod s)$,
$xy=(ab)^l\in <ab> = G$。故$H_1H_2\subseteq G$

故$H_1H_2 = G$

\subsection*{23}

对于无限群G,如果$G$中存在某个元素生成的循环群a是无限群,
则$\forall r\in Z^+, <a^r>\leq G$,因而G有无限多的子群

如果不存在这样的元素,那么我们可以每次从G中任取一个不属于
已经被取出的集合的元素,取出它生成的循环群,这是有限阶的。
因而我们取出的子群的元素总是有限的,总能取出不在这些里的元素,
进而能不断取出不同的子群。这个过程可以无限重复下去,因而G有无限多的子群

\subsection*{24}

(1)

\[
\sigma\tau = 
\begin{pmatrix}
    1 &2 &3 &4 &5\\
    3 &4 &5 &2 &1
\end{pmatrix}
\]

\[
\tau\sigma = 
\begin{pmatrix}
    1 &2 &3 &4 &5\\
    2 &4 &5 &1 &3
\end{pmatrix}
\]

\[
\sigma^{-1} = 
\begin{pmatrix}
    1 &2 &3 &4 &5\\
    2 &5 &4 &3 &1
\end{pmatrix}
\]

\[
\tau^{-1} = 
\begin{pmatrix}
    1 &2 &3 &4 &5\\
    3 &5 &2 &1 &4
\end{pmatrix}
\]

(2)

$\sigma = (1\ 5\ 2)(3\ 4) = (1\ 2)(1\ 5)(3\ 4)$

$\tau = (1\ 4\ 5\ 2\ 3) = (1\ 3)(1\ 2)(1\ 5)(1\ 4)$

\subsection*{25}

(1)

对任意$i\neq j, 1\notin\{i, j\}$,
$(1\ i)(1\ j)(1\ i) = (1\ i)(1\ i\ j) = (i\ j)$,
因而这个集合可以生成所有对换,而所有置换可以通过对换生成,
因而这个集合可以生成所有置换

(2)

由于对任意x, y, z, $(y\ z)(x\ y)(y\ z) = (x\ z)$

因而对任意$i < j$,在原来集合中,

$((i+1)\ (i+2))(i\ (i+1))((i+1)\ (i+2)) = (i\ (i+2))$

$((i+2)\ (i+3))(i\ (i+2))((i+2)\ (i+3)) = (i\ (i+3))$

依次类推,最终可以生成$(i\ (i+k)), \forall k\in N^+$,从而
可以生成$(i\ j)$

因而这个集合可以生成所有对换,而所有置换可以通过对换生成,
因而这个集合可以生成所有置换

\subsection*{27}

$(1\ 2\ 3\ 4)(1\ 2\ 3\ 4) = (1\ 3)(2\ 4)$

$(1\ 3)(2\ 4)(1\ 2\ 3\ 4) = (1\ 4\ 3\ 2)$

$(1\ 4\ 3\ 2)(1\ 2\ 3\ 4) = (1)$

因而,$H = \{(1\ 2\ 3\ 4), (1\ 3)(2\ 4), (1\ 4\ 3\ 2), (1)\}$

$H$在$S_4$中共有$|S_4| / |H| = 6$个右陪集,他们分别是:

$H = \{(1\ 2\ 3\ 4), (1\ 3)(2\ 4), (1\ 4\ 3\ 2), (1)\}$

$H(1\ 2) = \{(1\ 3\ 4), (1\ 4\ 2\ 3), (2\ 4\ 3), (1\ 2)\}$

$H(1\ 3) = \{(1\ 4)(2\ 3), (2\ 4), (1\ 2)(3\ 4), (1\ 3)\}$

$H(1\ 4) = \{(2\ 3\ 4), (1\ 2\ 4\ 3), (1\ 3\ 2), (1\ 4)\}$

$H(2\ 3) = \{(1\ 2\ 4), (1\ 3\ 4\ 2), (1\ 4\ 3), (2\ 3)\}$

$H(3\ 4) = \{(1\ 2\ 3), (1\ 3\ 2\ 4), (1\ 4\ 2), (3\ 4)\}$

\subsection*{28}

对于$x, y\in G, xH=yH\Leftrightarrow y^{-1}x\in H$

设$x = \begin{pmatrix}
    r_x& s_x\\
    0& 1
\end{pmatrix}, 
y = \begin{pmatrix}
    r_y& s_y\\
    0& 1
\end{pmatrix}$,
则
$y^{-1} = \begin{pmatrix}
    1/r_y& -s_y/r_y\\
    0& 1
\end{pmatrix}$,
$y^{-1}x = \begin{pmatrix}
    r_x/r_y& s_x / r_y - s_y/r_y\\
    0& 1
\end{pmatrix}$

$y^{-1}x\in H\Leftrightarrow r_x = r_y$

又$\begin{pmatrix}
    r& 0\\
    0& 1
\end{pmatrix}\begin{pmatrix}
    1& t\\
    0& 1
\end{pmatrix} = \begin{pmatrix}
    r& t\\
    0& 1
\end{pmatrix}$

故H在G中的全部左陪集形如$\{
    \begin{pmatrix}
        r& t\\
        0& 1
    \end{pmatrix} | t\in Q
\}, r\neq 0, r\in Q$

\subsection*{31}

假设题设中的$p^m$阶群为G

对m作归纳法:

当m=1时,G本身为自己的p阶子群

当$m\geq 2$时,假设命题对所有小于m的正整数成立,那么,
G中一定存在不为单位元的元素$a$,$<a>$构成G的子群。若
$|a| = p^{m}$,则$<a^{p^{m-1}}>$构成G的p阶子群。
否则,由于$|a|\Big|p$,知$|a| = p^{k}, k < m$,又
$a\neq e$,知$k \neq 0$,因而由归纳假设,$<a>$有p阶
子群

由归纳原理,命题得证

\subsection*{34}

我们记$f(\sigma)$为$\sigma$的轮换指数,记$g((a_1\ a_2\ ... a_n)) = \{a_1, a_2, ..., a_n\}$

我们希望证明,对任意置换$\sigma$和任意
$a\neq b$,$f(\sigma (a\ b)) = f((a\ b)\sigma)$

我们设$\sigma = \sigma_1\sigma_2...\sigma_n$为$\sigma$的
不交轮换分解。我们设下标最大的零个、一个或两个轮换中的元素与$A=\{a, b\}$有交

我们分类讨论这些情况:

(1) 若仅有$\sigma_n$与$A$有交,那么:

i. 若$|g(\sigma_n)\bigcap A| = 1$,可不妨设$\sigma_n = (a\ a_2\ ... a_n)$,则
$\sigma_n(a\ b) = (a\ b\ a_2\ a_3\ ...a_n)$,
$(a\ b)\sigma_n = (a\ a_2\ a_3\ ...\ a_n\ b)$,
这两项都与其他轮换不交,长度相等,故$f(\sigma (a\ b)) = f(\sigma_1\sigma_2...\sigma_n(a\ b)) = f(\sigma_1\sigma_2...\sigma_{n-1}(a\ b)\sigma_n) = f((a\ b)\sigma_1\sigma_2...\sigma_n) = f((a\ b)\sigma)$

ii. 若$|g(\sigma_n)\bigcap A| = 2$,且对应的两个元素在$\sigma_n$中相连,可不妨设$\sigma_n = (a\ b\ a_3\ ... a_n)$,有
$\sigma_n(a\ b) = (a\ a_3\ ...a_n)$,
$(a\ b)\sigma_n = (b\ a_3\ ...a_n)$,
这两项都与其他轮换不交,长度相等,故$f(\sigma (a\ b)) = f(\sigma_1\sigma_2...\sigma_n(a\ b)) = f(\sigma_1\sigma_2...\sigma_{n-1}(a\ b)\sigma_n) = f((a\ b)\sigma_1\sigma_2...\sigma_n) = f((a\ b)\sigma)$

iii. 若$|g(\sigma_n)\bigcap A| = 2$,且对应的两个元素在$\sigma_n$中不相连,可不妨设$a_1 = a, a_i = b(i\neq 2\land i\neq n)$,有
$\sigma_n(a\ b) = (a\ a_{i+1}\ a_{i+2}\ ... a_n)(b\ a_2\ a_3\ ... a_{i-1})$,
$(a\ b)\sigma_n = (a\ a_2\ a_3\ ...a_{i-1})(b\ a_{i+1}...a_n)$
这四项与其他轮换不交,且前者的前一部分和后者的后一部分等长,后者的前一部分和前者的后一部分等长,故
$f(\sigma (a\ b)) = f(\sigma_1\sigma_2...\sigma_n(a\ b)) = f(\sigma_1\sigma_2...\sigma_{n-1}(a\ b)\sigma_n) = f((a\ b)\sigma_1\sigma_2...\sigma_n) = f((a\ b)\sigma)$

(2) 若A与$\sigma_{n-1}$与$\sigma_n$都相交,不妨设$\sigma_n = (a\ a_2\ ... a_n)$,$\sigma_{n-1} = (b\ b_2\ ... b_n)$,
则$\sigma_{n-1}\sigma_n(a\ b) = \sigma_{n-1}(b\ a_2\ ...\ a_n\ a) = (b\ a_2\ ...\ a_n\ a\ b_2\ b_3\ ...\ b_n)$,
$(a\ b)\sigma_{n-1}\sigma_n = (a\ b\ b_2\ ... b_n)\sigma_n = (a\ a_2\ ... a_n\ b\ b_2\ ...b_n)$,
这两项都与其他轮换不交,长度相等,故$f(\sigma (a\ b)) = f(\sigma_1\sigma_2...\sigma_n(a\ b)) = f(\sigma_1\sigma_2...\sigma_{n-2}(a\ b)\sigma_{n-1}\sigma_n) = f((a\ b)\sigma_1\sigma_2...\sigma_n) = f((a\ b)\sigma)$

(3) 若没有一项与A有交,那么这些轮换都可交换,进而结论成立。

由此,对于$\tau^{-1}\sigma\tau$,我们设$\tau = \tau_1\tau_2...\tau_n$为$\tau$的轮换分解,有$f(\tau^{-1}\sigma\tau) = f(\tau_n\tau_{n-1}...\tau_1\sigma\tau_1\tau_2...\tau_n) = f(\tau_n\tau_n\tau_{n-1}...\tau_1\sigma\tau_1\tau_2...\tau_{n-1})=f(\tau_{n-1}...\tau_1\sigma\tau_1\tau_2...\tau_{n-1})=f(\tau_{n-2}...\tau_1\sigma\tau_1\tau_2...\tau_{n-2}) = ... = f(\sigma)$

因而此题得证。

\end{document}