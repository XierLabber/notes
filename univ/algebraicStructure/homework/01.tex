\documentclass[UTF8]{ctexart}
\usepackage{amsmath}
\usepackage{forest}
\usepackage{algorithm}
\usepackage{algorithmic}
\usepackage{tabularx}
\title{\vspace{-4cm}2023代组第一次作业}
\author{2000013058 杨仕博}
\date{\today}
\begin{document}
\maketitle

\subsection*{7(2)}

(1)

$a, b\in Z^+ \Rightarrow \min(a, b) = a\ or\ b\in Z^+$,
因而$\circ$是$Z^+$上的二元运算,故它构成代数系统。

由$a\circ b = \min(a, b) = \min(b, a) = b\circ a$,
$a\circ b \circ c = \min(\min(a, b), c) = \min(a, b, c)$
$= \min(a, \min(b, c)) = a\circ (b\circ c)$,
$a\circ a = \min(a, a) = a$知它符合交换律、结合律和幂等律

(2)

单位元:若b为单位元,则

\[
\begin{aligned}
    &\forall a, a\circ b = b\circ a = a\\
    \Leftrightarrow &\forall a, \min(a, b) = a\\
    \Rightarrow &\min(b, b + 1) = b + 1\\
    \Rightarrow &b \geq b + 1
\end{aligned}    
\]

矛盾!故此时不存在单位元

零元:若b为零元,则

\[
\begin{aligned}    
    &\forall a, a\circ b = b\circ a = b\\
    \Leftrightarrow &\forall a, \min(a, b) = b\\
    \Leftrightarrow &\forall a, b\leq a\\
    \Leftrightarrow b = 1
\end{aligned}    
\]

故零元为1

\subsection*{7(4)}

令$a = 1, b = 2$,有$a\circ b = 0.5 + 2 = 2.5\notin Z^+$,
因而这不是一个代数系统。

\subsection*{9}

(1)

$x * y = x + y - xy = y + x - yx = y * x$,因而满足交换律。

\[
\begin{aligned}
    x * y * z &= (x + y - xy) * z\\
              &= x + y - xy + z - zx -zy + xyz
\end{aligned}    
\]

\[
\begin{aligned}
    x * (y * z) &= x * (y + z - yz)\\
                &= x + y + z - yz - xy - xz + xyz
\end{aligned}    
\]

故$x * y * z = x * (y * z)$,因而满足结合律。

令$x = 2$,有$x * x = 2 + 2 - 4 = 0 \neq 2 = x$,
因此不满足幂等律

综上,满足交换律和结合律,不满足幂等律

(2)

单位元:若e为单位元,则

\[
\begin{aligned}
    &\forall a, a * e = e * a = a\\
    \Leftrightarrow &\forall a, a + e - ae = a\\
    \Leftrightarrow &\forall a, e(1 - a) = 0\\
    \Leftrightarrow &e = 0
\end{aligned}
\]

故单位元为0。

零元:若z为零元,则

\[
\begin{aligned}
    &\forall a, a * z = z * a = z\\
    \Leftrightarrow &\forall a, a + z - az = z\\
    \Leftrightarrow &\forall a, a(1 - z) = 0\\
    \Leftrightarrow &z = 1
\end{aligned}    
\]

故零元为1。

逆元:若x是可逆元,y是其逆元,则

\[
\begin{aligned}
    &x * y = y * x = 0\\
    \Leftrightarrow & x + y - xy = 0\\
    \Leftrightarrow & x = y(x - 1)
\end{aligned}
\]

若$x\neq 1$,则$y = \frac{x}{x - 1}$为x的逆元。

\subsection*{11}

(1)

交换律:

$<0, 1>\circ<1, 1> = <0, 1>$,而$<1, 1>\circ<0, 1> = <0, 2>$,
因此不满足交换律。

结合律:

\[
\begin{aligned}
    &<a, b>\circ<c, d>\circ<e, f>\\
    =& <ac, ad + b>\circ<e, f>\\
    =& <ace, acf + ad + b> 
\end{aligned}    
\]

\[
\begin{aligned}
    &<a, b>\circ(<c, d>\circ<e, f>)\\
    =& <a, b>\circ(<ce, cf + d>)\\
    =& <ace, acf + ad + b>
\end{aligned}
\]

因此满足结合律。

(2)

单位元:

若$<e, f>$为单位元,则

\[
\begin{aligned}
    &\forall <a, b>, <a, b> \circ <e, f> = <e, f> \circ <a, b> = <a, b>\\
    \Leftrightarrow &\forall <a, b>, <ae, af + b> = <a, b> \land <ae, eb + f> = <a, b>\\
    \Leftrightarrow &e = 1\land f = 0
\end{aligned}
\]

故单位元为<1, 0>。

零元:

若$<y, z>$为零元,则

\[
\begin{aligned}
    &\forall <a, b>, <a, b> \circ <y, z> = <y, z> \circ <a, b> = <y, z>\\
    \Leftrightarrow &\forall <a, b>, <ay, az + b> = <y, z> \land <ay, by + z> = <y, z>
\end{aligned}    
\]

令$<a, b>$分别取<0, 1>和<0, 2>,我们有<0, 1> = <y, z>且<0, 2> = <y, z>,
矛盾!故没有零元。

逆元:

若$<a, b>$可逆,$<c, d>$是其逆元,有

\[
\begin{aligned}
    &<a, b>\circ <c, d> = <ac, ad + b> = <1, 0>\\
    \Leftrightarrow &ac = 1 \land ad + b = 0\\
    \Leftrightarrow &a\neq 0 \land c = \frac{1}{a} \land d = -\frac{b}{a}
\end{aligned}    
\]

故$<a, b>$有逆元当且仅当$a \neq 0$,其逆元为$<\frac{1}{a}, -\frac{b}{a}>$

\subsection*{16}

(1)

\begin{tabularx}{\linewidth}{|X|X|X|X|X|X|X|}
\hline
        & <0, 0> & <0, 1> & <1, 0> & <1, 1> & <2, 0> & <2, 1>\\
\hline
<0, 0>  & <0, 0> & <0, 1> & <1, 0> & <1, 1> & <2, 0> & <2, 1>\\
\hline
<0, 1>  & <0, 1> & <0, 0> & <1, 1> & <1, 0> & <2, 1> & <2, 0>\\
\hline
<1, 0>  & <1, 0> & <1, 1> & <2, 0> & <2, 1> & <0, 0> & <0, 1>\\
\hline
<1, 1>  & <1, 1> & <1, 0> & <2, 1> & <2, 0> & <0, 1> & <0, 0>\\
\hline
<2, 0>  & <2, 0> & <2, 1> & <0, 0> & <0, 1> & <1, 0> & <1, 1>\\
\hline
<2, 1>  & <2, 1> & <2, 0> & <0, 1> & <0, 0> & <1, 1> & <1, 0>\\
\hline
\end{tabularx}

(2)

单位元:

若$<a, b>$是单位元,则

\[
\begin{aligned}
    &\forall <x, y>, <a, b> \oplus <x, y> = <x, y> \oplus <a, b> = <x, y>\\
    \Leftrightarrow &(x + a)\% 3 = x\land (y + b)\%2 = y\\
    \Leftrightarrow &<a, b> = <0, 0>
\end{aligned}    
\]

故单位元为<0, 0>

逆元:

若$<a, b>$可逆,其逆元为$<c, d>$,则

\[
\begin{aligned}
    &<a, b> + <c, d> = <0, 0>\\
    \Leftrightarrow &<c, d> = <3 - a, 2 - b>
\end{aligned}    
\]

故所有元素都可逆,且$<a, b>$的逆元是$<3 - a, 2 - b>$

\subsection*{18}

我们构造如下映射:$f(c) = \begin{pmatrix}
    a & b\\
    -b & a
\end{pmatrix}$,其中,$c = a + bi$

首先,我们证明f是同态映射:

对复数$x = a + bi, y = c + di (a, b, c, d\in R)$,
$f(x + y) = f((a + c) + (b + d)i) = \begin{pmatrix}
    a + c & b + d\\
    -(b + d) & (a + c)
\end{pmatrix} = \begin{pmatrix}
    a & b\\
    -b & a
\end{pmatrix} + \begin{pmatrix}
    c & d\\
    -d & c
\end{pmatrix} = f(x) + f(y)$

$f(x)\cdot f(y) = \begin{pmatrix}
    a & b\\
    -b & a
\end{pmatrix}\cdot \begin{pmatrix}
    c & d\\
    -d & c
\end{pmatrix} = \begin{pmatrix}
    ac-bd & ad+bc\\
    -(ad+bc) & ac-bd
\end{pmatrix} = f(ac-bd + (ad + bc)i) = 
f((a + bi)\cdot (c + di)) = f(xy)$

因此,f是同态映射。

接下来,我们证明f是双射。

对任意$Y = \begin{pmatrix}
    a & b\\
    -b & a
\end{pmatrix}\in B$,$f(a+bi) = Y$,因而f是满射;

对任意$x, y\in C$,若$f(x) = f(y)$,则$Re(x) = Re(y)\land
Im(x) = Im(y)$,从而$x = y$,因而f是单射。

故f是双射,从而是同构映射。

因而$V_1$同构于$V_2$

\subsection*{24(2)}

这是同态,因为对$x, y\in C, f(xy) = |xy| = |x||y| = f(x)f(y)$

对任意实数$x\geq 0, \phi(x) = x$,而对任意$x\in C, \phi(x) \geq 0$,
因而同态像是$\{0\}\bigcup R^+$

\subsection*{24(4)}

$\phi$不是同态,因为$\phi(1\cdot 1) = \phi(1) = 2, 
phi(1)\cdot \phi(1) = 4$

\subsection*{27(2)}

不是。这不是一个等价关系,$|1 - 5| < 5\Rightarrow 1R5, 
|5 - 9| < 5\Rightarrow 5R9$,但是$|1 - 9| > 5$,即这
个二元关系不具备传递性,继而不是等价关系

\subsection*{27(4)}

不是。这不是一个等价关系,$5\geq 4\Rightarrow 5R4$,但是$4<5$,
$4R5$不成立,
因而这个而元关系不对称,继而不是等价关系

\end{document}