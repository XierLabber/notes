\documentclass[UTF8]{ctexart}
\usepackage{amsmath}
\usepackage{forest}
\usepackage{algorithm}
\usepackage{algorithmic}
\usepackage{tabularx}
\usepackage{booktabs}
\title{\vspace{-4cm}2023代组第六次作业}
\author{2000013058 杨仕博}
\date{\today}
\begin{document}
\maketitle

\subsection*{38}

(1)

$G = \{e, a, a^2, a^3\}$

$\forall x\in G,\ assume\ x = a^j, \forall i\in Z, a^ixa^{-i} = a^{i+j-i} = a^j = x\Rightarrow$
G的所有共轭类为$\{e\}, \{a\}, \{a^2\}, \{a^3\}$

(2)

设$G = \{e, a, b, c\}$

$\bar{e} = \{e\}$

$\bar{a} = \{eae, aaa, bab, cac\} = \{a, a, a, a\} = \{a\}$

同理$\bar{b} = \{b\}, \bar{c} = \{c\}$

G的所有共轭类为$\{e\}, \{a\}, \{b\}, \{c\}$

\subsection*{41}

$k = |\bar{a}| = [G: N(a)] = |G| / |N(a)| \Rightarrow|N(a)| = n/k$

又$C\leq N(a)\Rightarrow c \Big| |N(a)|\Rightarrow c\Big| n / k\Rightarrow k\Big| n / c$

\subsection*{42}

设$G = <a>$为循环群,其任意子群必然形如$<a^i>$

对任意$x\in G, y\in <a^i>,\ assume x = a^j, y = a^{is}, xyx^{-1} = a^ja^{is}a^{-j} = a^{j + is - j} = a^{is} = y \in <a^i>$,

故$<a^i>$为正规子群

\subsection*{46}

(1)

首先证明H是G的子群,明显I在H中,故而H非空:$\forall A\in H, B\in H, |AB^{-1}| = \frac{|A|}{|B|} > 0\Rightarrow AB^{-1}\in H$

其次证明H的正规性:$\forall X\in G, A\in H, |X|\neq 0, |XAX^{-1}| = \frac{|X||A|}{|X|} = |A|\Rightarrow XAX^{-1}\in H$

故H是G的正规子群

(2)

$\forall A, B\in G, AH = BH\Leftrightarrow BA^{-1}\in H\Leftrightarrow \frac{|B|}{|A|} > 0\Leftrightarrow Sgn(|B|) = Sgn(|A|)$

又$\forall A\in G, |A| > 0\lor |A| < 0$,故$[G:H] = 2$

\subsection*{47}

(1)

$\forall A, B\in M_n(Q), \phi(A)\phi(B) = |A||B| = |AB| = \phi(AB)$

故$\phi$是同态映射

(2)

$\forall A\in G_1, \phi(A) = |A|\in Q^*\Rightarrow \phi(G_1)\subseteq Q^*$

$\forall x\in Q^+, |diag\{x, I_{n-1}\}| = x, |diag\{-x, I_{n-1}\}| = -x\Rightarrow Q^*\subseteq \phi(G_1)$

故$\phi(G_1) = Q^*$

$A\in ker\phi\Leftrightarrow \phi(A) = 1\Leftrightarrow |A| = 1$

故$ker\phi$是所有行列式为1的$M_n(Q)$中的矩阵构成的集合

\subsection*{48}

若$\phi$是$<Q, +>$到$<Z, +>$的同态映射,且$\exists a\in Q, s.t. \phi(a)\neq 0$,不妨设$\phi(a) = s\in Z$

由$\phi(0) = 0$知$a\neq 0$,我们考察$\phi(a / 2s^2)$

由于$2s^2\phi(a / 2s^2) = \phi(2s^2(a / 2s^2)) = \phi(a) = s$,有$\phi(a / 2s^2) = \frac{1}{2s}\notin Z$,与群的定义矛盾!

故对于任意$<Q, +>$到$<Z, +>$的同态映射$\phi$,不存在$a\in Q, s.t. \phi(a)\neq 0$,故$\phi$为零同态,命题得证

\subsection*{51}

(1)

首先$e\in \phi^{-1}(H)$,故而$\phi^{-1}(H)$非空

又$\forall x, y\in \phi^{-1}(H), \phi(xy^{-1}) = \phi(x)\phi(y)^{-1}$

且$\phi(x)\in H, \phi(y)\in H\Rightarrow \phi(y)^{-1}\in H\Rightarrow \phi(x)\phi(y)^{-1}\in H\Rightarrow xy^{-1}\in \phi^{-1}(H)$

故$\phi^{-1}(H)$构成$G_1$的子群

(2)

若H是$G_2$的正规子群,由(1),$\phi^{-1}(H)$构成$G_1$的子群

$\forall x\in G_1, y\in \phi^{-1}(H), \phi(xyx^{-1}) = \phi(x)\phi(y)\phi(x)^{-1}$

又$H$是$G_2$的正规子群$\Rightarrow\phi(x)\phi(y)\phi(x)^{-1}\in H\Rightarrow xyx^{-1}\in \phi^{-1}(H)$,由x,y的任意性,$\phi^{-1}(H)$是$G_1$的正规子群

\end{document}