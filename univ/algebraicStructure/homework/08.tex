\documentclass[UTF8]{ctexart}
\usepackage{amsmath}
\usepackage{forest}
\usepackage{algorithm}
\usepackage{algorithmic}
\usepackage{tabularx}
\usepackage{booktabs}
\title{\vspace{-4cm}2023代组第八次作业}
\author{2000013058 杨仕博}
\date{\today}
\begin{document}
\maketitle

\subsection*{20}

(1)

$\forall a, a'\in A, b, b'\in B, (a + b) - (a' + b') = (a + (-a')) + (b + (-b'))\in A + B$

故A+B构成R的加法的子群

$\forall a\in A, b\in B, (a + b)(A + B) = aA + aB + bA + bB$

由于$aA\subseteq A, aB\subseteq B, bA\subseteq A, bB\subseteq B\Rightarrow aA + aB + bA + bB\subseteq A + B$,

同理$(A + B)(a + b)\subseteq A + B$

故A+B构成R的理想

(2)

考察四元数集合$R_0 = \{a + bi + cj + dk | a, b, c, d\in R\}$(后一个R指实数域)

$A, B$是$R_0$的两个不同方向的复数子环:$A = \{a + bi | a, b\in R\}, B = \{a + cj | a, c\in R\}$

那么,$i\in A + B, j\in A + B$

但是$ij = k \notin A + B$,因而$A + B$在乘法上不构成半群,进而$A + B$不是$R_0$的子环

\subsection*{23}

(1)

$\forall x, y\in Z, 4x + 4y = 4(x + y)\in D$

$\forall x\in Z, D(4x) = 4xD = \{16xy | y\in Z\}\subseteq D$

因而D是A的一个理想

(2)

$\forall x, y\in A, x + D = y + D\Leftrightarrow x - y\in D\Leftrightarrow x\equiv y (\mod 4)$

故$A / D = \{D, 2 + D\}$

\subsection*{26}

(1)

若H是R的极大理想:

考察$f: R\rightarrow R/H, f(x) = x + H$,那么,明显f为满射,
并且$\forall x, y\in R, f(x + y) = x + H + (y + H) = (x + y) + H = f(x + y)$,
$f(xy) = xy + H = xy + xH + Hy + H = (x + H)(y + H) = f(x)f(y)$(利用了H的理想性质),
故$f$为同态映射,因此$R / H$为环。又同态保幺元和交换,因而$R / H$含幺且可交换。

下面只需要证明$R / H$中除零元(由上,零元为$f(0) = H$)外的每个元素都可逆:

$\forall a\in R, a\notin H, $令$A = \{h + ax | h\in H, x\in R\}$,
我们证明,$A=R$

由于$\forall h_1, h_2\in H, x_1, x_2\in R, (h_1 + ax_1) - (h_2 + ax_2) = (h_1 - h_2) + a(x_1 - x_2)\in A$,
$\forall r\in R, h\in H, x\in R, r(h + ax) = rh + rax\in A, (h + ax)r = hr + axr\in A$,
因而A是R的理想,并且$H\subset A$,由H的最大性,$A = R$

故而$1\in A, \exists b\in R, s.t. ab\in 1 + H$

故$(a + H)(b + H) = ab + H = 1 + H$

于是$R / H$为域

(2)

如果R/H是域:

对于R的理想$A$,若$H\subset A$,$\exists a\in A, a\notin H$

那么,$\forall s\in R$,$\exists b\in R, s.t.\ (a + H)(b + H) = s + H$(因为$R / H$是域)$\Rightarrow ab = s + H\Rightarrow s \in A$,由s的任意性,$R\subseteq A$,故A=S

\subsection*{30}

(1)

$\forall c\in F, g(x) \equiv c\in F[x], \phi(g) = c$

故$\phi$为满射

$\forall f(x), g(x)\in F[x], \phi(f + g) = (f + g)(0) = f(0) + g(0) = \phi(f) + \phi(g), \phi(fg) = fg(0)$

设$f(x) = \sum\limits_{i = 0}^n a_ix^i, g(x) = \sum\limits_{i = 0}^m b_ix^i$,则$\phi(fg) = fg(0) = a_0b_0 = f(0)g(0) = \phi(f)\phi(g)$,因而$\phi$是满同态

(2)

$f\in ker\phi \Leftrightarrow f(0) = 0\Leftrightarrow x | f(x)$

故$ker\phi = \{f\in F[x] \Big| x \big| f(x)\}$

$F[x] / ker\phi = \{g + ker\phi\Big| g\in F[x]\} = \{c + ker\phi \Big| c\in F\}$

\subsection*{34}

i.

首先证明EndG关于+构成Abel群:

(1) 证明EndG关于+封闭(广群):

$\forall f, g\in EndG, x, y\in G, (f + y)(x + y) = f(x + y) + g(x + y) = f(x) + f(y) + g(x) + g(y) = (f + g)(x) + (f + g)(y)$

(2) 证明EndG关于+成立结合律(半群):

$\forall f, g, h\in EndG, x\in G, (f + (g + h))(x) = f(x) + (g + h)(x) = f(x) + g(x) + h(x) = (f + g + h)(x)$

(3) 证明EndG关于+存在单位元(独异点):

明显$e:G\rightarrow G, e(x) = 0$满足$e\in EndG$

$\forall f\in EndG, x\in G, (f + e)(x) = f(x) + e(x) = f(x), (e + f)(x) = e(x) + f(x) = f(x)$

故$e$是单位元

(4) 证明每个EndG中的元素含有逆元(群):

$\forall f\in EndG, $明显G上的映射g, $\forall x\in G, g(x) = -f(x)$满足g是G上的自同态,并且$\forall x\in G, (f + g)(x) = f(x) + g(x) = 0 = e(x)$

故EndG构成群

(5) 证明EndG的加法可以交换(Abel群):

$\forall f, g\in EndG, x\in G, (f + g)(x) = f(x) + g(x) = g(x) + f(x) = (g + f)(x)$

其次证明EndG关于$\circ$构成半群:

(1) 证明EndG关于$\circ$封闭:

$\forall f, g\in EndG, x, y\in G, f\circ g(x) = f(g(x))\in G, f\circ g(x + y) = f(g(x + y)) = f(g(x) + g(y)) = f(g(x)) + f(g(y)) = f\circ g(x) + f\circ g(y)$

故$f\circ g\in EndG$

(2) 证明EndG关于$\circ$满足结合律:

$\forall f, g, h\in EndG, x\in G, f\circ (g\circ h)(x) = f(g\circ h(x)) = f(g(h(x))) = f\circ g\circ h(x)$

故EndG关于$\circ$构成半群

最后证明分配律成立:

$\forall f, g, h\in EndG, x\in G, f\circ(g + h)(x) = f((g + h)(x)) = f(g(x) + h(x)) = (f\circ g)(x) + (f\circ h)(x) = (f\circ g + f\circ h)(x)$

同理$(g + h)\circ f = g\circ f + h\circ f$

故$<EndG, +, \circ>$构成环

ii.

对于在G自身上的映射f,

f为同态$\Leftrightarrow$
$\forall i, j\in Z_n, f(ia + ja) = f(ia) + f(ja)$
$\Rightarrow \forall i\in Z_n, f(ia) = if(a)$

而$\forall i\in Z_n, f(ia) = if(a) \Rightarrow \forall$
$i, j\in Z_n, f(ia + ja) = f((i + j)a) = (i + j)f(a) = if(a) + jf(a)$

故所有同态函数f必然满足$f(ia) = if(a)$,且所有这样的f均为同态函数,
故G的自同态环为

$<A, +, \circ>$,其中$A = \{f_p | \forall i\in Z_n, f_p(ia) = ipa, p \in Z_n\}$

\end{document}