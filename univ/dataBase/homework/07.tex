\documentclass[UTF8]{ctexart}
\usepackage{amsmath}
\usepackage{forest}
\usepackage{algorithm}
\usepackage{algorithmic}
\usepackage{graphicx}
\usepackage{tabularx}
\title{\vspace{-4cm}2023数据库概论第七章作业}
\author{2000013058 杨仕博}
\date{\today}
\begin{document}
\maketitle

\subsection{}

左:A
右:
双:BCDE

1. 在所有原子属性中,左部属性只有A,剩下的全部为双部属性。

$A_F^+ = A$

$(AB)_F^+ = ABCDE$

$(AC)_F^+ = ABCDE$

$(AD)_F^+ = AD$

$(AE)_F^+ = AE$

因为$AB$是候选码,$ABC, ABD, ABE$不是候选码

因为$AC$是候选码,$ACD, ACE$不是候选码

$(ADE)_F^+ = ADE$

因为$AB, AC$是候选码,$ABCD, ABCE, ABDE, ACDE, ABCDE$不是候选码

因此候选码为$AB$和$AC$

2. 

该关系中

(1) 每一分量不可再分,满足1NF

(2) 非主属性有D、E,而D只部分依赖于B(而非AB),因而不满足第二范式。

故范式级别为1NF

3. 

\end{document}