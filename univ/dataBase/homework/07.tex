\documentclass[UTF8]{ctexart}
\usepackage{amsmath}
\usepackage{forest}
\usepackage{algorithm}
\usepackage{algorithmic}
\usepackage{graphicx}
\usepackage{tabularx}
\usepackage{listings}
\title{\vspace{-4cm}2023数据库概论第七章作业}
\author{2000013058 杨仕博}
\date{\today}
\begin{document}
\maketitle

\subsection{}

左:A

右:

双:BCDE

1. 在所有原子属性中,左部属性只有A,剩下的全部为双部属性。

$A_F^+ = A$

$(AB)_F^+ = ABCDE$

$(AC)_F^+ = ABCDE$

$(AD)_F^+ = AD$

$(AE)_F^+ = AE$

因为$AB$是候选码,$ABC, ABD, ABE$不是候选码

因为$AC$是候选码,$ACD, ACE$不是候选码

$(ADE)_F^+ = ADE$

因为$AB, AC$是候选码,$ABCD, ABCE, ABDE, ACDE, ABCDE$不是候选码

因此候选码为$AB$和$AC$

2. 

该关系中

(1) 每一分量不可再分,满足1NF

(2) 非主属性有D、E,而D只部分依赖于B(而非AB),因而不满足第二范式。

故范式级别为1NF

3. 

首先我们求出F的最小覆盖

首先,F不需要进行单属性化

之后,我们对F进行既约化

对$AB\rightarrow C$,有$A^+ = A$,故不能删去B。

对$AB\rightarrow C$,有$B^+ = BD$,故不能删去A。

对$CD\rightarrow E$,有$C^+ = C$,故不能删去D。

对$CD\rightarrow E$,有$D^+ = D$,故不能删去C。

对$CE\rightarrow B$,有$C^+ = C$,故不能删去E。

对$CE\rightarrow B$,有$E^+ = E$,故不能删去C。

对$AC\rightarrow B$,有$A^+ = A$,故不能删去B。

对$AC\rightarrow B$,有$C^+ = C$,故不能删去A。

故F无需进行既约化

之后,我们对F进行无冗余化

去掉$AB\rightarrow C$,我们有$(AB)^+ = ABD$,故不能去掉此依赖。

去掉$B\rightarrow D$,我们有$B^+ = B$,故不能去掉此依赖。

去掉$CD\rightarrow E$,我们有$(CD)^+ = CD$,故不能去掉此依赖。

去掉$CE\rightarrow B$,我们有$(CE)^+ = CE$,故不能去掉此依赖。

去掉$AC\rightarrow B$,我们有$(AC)^+ = AC$,故不能去掉此依赖。

故F无需进行无冗余化。

之后,我们进行保持无损连接的分解。

$B\rightarrow D$中,B不是主码,把$\{BD\}$分解出来,得到$\{ABCE, BD\}$

$CE\rightarrow B$中,CE不是主码,把$\{BCE\}$分解出来,得到$\{ACE, BCE, BD\}$

对$R_1(A, C, E), F_1 = \{AC\rightarrow E\}$

对$R_2(B, C, E), F_2 = \{CE\rightarrow B\}$

对$R_3(B, D), F_3 = \{B\rightarrow D\}$

此时,没有一个依赖关系左侧不是码

之后,我们进行保持函数依赖的分解。

我们直接按照函数关系把原数据库关系分解为$\{ABC, BD, CDE, BCE\}$

对$R_1(A, B, C), F_1 = \{AB\rightarrow C, AC\rightarrow B\}$

对$R_2(B, D), F_2 = \{B\rightarrow D\}$

对$R_3(C, D, E), F_3 = \{CD\rightarrow E\}$

对$R_4(B, C, E), F_4 = \{CE\rightarrow B\}$

\subsection{}

$A_F^+ = A$

$B_F^+ = B$

$C_F^+ = C$

$D_F^+ = AD$

$(AB)_F^+ = ABD$

$(AC)_F^+ = ABCDE$

$(AD)_F^+ = AD$

$(BC)_F^+ = ABCDE$

$(BD)_F^+ = ABD$

$(CD)_F^+ = ABCDE$

$(ABD)_F^+ = ABD$

所以投影为$\{D\rightarrow A, $
$AB\rightarrow D, AC\rightarrow $
$B, AC\rightarrow $
$D, BC\rightarrow A, BC\rightarrow $
$D\}$

\subsection{}

首先我们求这个关系模式的候选码

左:B

右:H

双:C, D, F, G

$B^+ = B$

$(BC)^+ = BCDFGH$

$(BD)^+ = BD$

$(BF)^+ = BFD$

$(BG)^+ = BCDFGH$

故而候选码为$BC, BG$

接下来我们求依赖集的最小覆盖

首先单属性化:

$\{
    BG\rightarrow C, 
    BG\Rightarrow D, 
    G\rightarrow F, 
    CD\rightarrow G, 
    CD\rightarrow H, 
    C\Rightarrow F, 
    C\rightarrow G, 
    F\rightarrow D
\}$

之后既约化:

$BG\rightarrow C$中,$B^+ = B$, $G^+ = GFD$, 故不能删

$BG\Rightarrow D$中,$B^+ = B, G^+ = GFD$, 故可以变为$G\rightarrow D$

$CD\rightarrow G$中,$C^+ = CFGDH$, 故可以变为$C\rightarrow G$

$CD\rightarrow H$同理可以变为$C\rightarrow H$

现在依赖关系变为

$\{
    BG\rightarrow C, 
    G\rightarrow D, 
    G\rightarrow F, 
    C\rightarrow G, 
    C\rightarrow H, 
    C\Rightarrow F, 
    F\rightarrow D
\}$

之后无冗余化:

去掉$BG\rightarrow C$后,$(BG)^+ = BGDF$,故不能删

去掉$G\rightarrow D$后,$G^+ = GFD$,故可以删

变为

$\{
    BG\rightarrow C, 
    G\rightarrow F, 
    C\rightarrow G, 
    C\rightarrow H, 
    C\Rightarrow F, 
    F\rightarrow D
\}$

去掉$G\rightarrow F$后,$G^+ = G$,故不能删

去掉$C\rightarrow G$后,G不依赖于任何东西,故不能删

去掉$C\rightarrow H$后,H不依赖于任何东西,故不能删

去掉$C\rightarrow F$后,$C^+ = CGHFD$,故可以删

变为

$\{
    BG\rightarrow C, 
    G\rightarrow F, 
    C\rightarrow G, 
    C\rightarrow H,  
    F\rightarrow D
\}$

去掉$F\rightarrow D$后,$F^+ = F$,故不能删

于是最小覆盖为

$\{
    BG\rightarrow C, 
    G\rightarrow F, 
    C\rightarrow G, 
    C\rightarrow H,  
    F\rightarrow D
\}$

故分解为$\rho = \{BCG, FG, CG, CH, FD\}$

对$R_1(B, C, G), F_1 = \{BG\rightarrow C\}$

对$R_2(F, G), F_2 = \{G\rightarrow F\}$

对$R_3(C, G), F_3 = \{C\rightarrow G\}$

对$R_4(C, H), F_1 = \{C\rightarrow H\}$

对$R_5(F, D), F_1 = \{F\rightarrow D\}$

\subsection{}

1NF显然满足

码只有A,故2NF满足

由于可能成立$BD\rightarrow C$,故会产生$A\rightarrow BD, BD\rightarrow C$,
不满足3NF

故R的范式级别为2NF

\subsection{}

关系代数:

$\pi_{AB}(R)\bowtie \pi_{AC}(R) = R$

sql语句:

\begin{lstlisting}[language=sql]
SELECT EXISTS(
    SELECT * FROM 
    (
        SELECT count(*) FROM
        (
            SELECT DISTINCT r1.A, r2.B, r1.C 
            FROM r AS r1
            FULL JOIN r AS r2
            ON r1.A = r2.A
        ) 
    ) AS tb1
    JOIN
    (
        SELECT count(*) FROM r
    ) AS tb2
) AS result
\end{lstlisting}

\subsection{}

首先不需要单属性化

之后进行既约化

$(AC)^+ = ABCDE$,$A^+ = AB$,$C^+ = C$

故可以删为

$\{
    AC\rightarrow E,
    E\rightarrow D,
    A\rightarrow B,
    AC\rightarrow D
\}$

明显其余部分不能删减

其次进行无冗余化

$AC\rightarrow D$可以去掉,因为去掉之后$(AC)^+ = ACED$

$A\rightarrow B$若去掉,则B不依赖任何值,故不能删

$E\rightarrow D$若去掉,则D不依赖任何值,故不能删

故最小覆盖为

$\{
    AC\rightarrow E,
    E\rightarrow D,
    A\rightarrow B
\}$

\end{document}